\documentclass{article}
\title{Numerical Methods - Project 1}
\author{Thomas Simcox (Group 18)}
\date{October, 2021}

\usepackage{amsmath, amsfonts, amsthm, amssymb, fancyvrb, listings, hyperref}
\usepackage[margin=1in]{geometry}
\hypersetup{
    colorlinks=true,
    linkcolor=blue
}

\begin{document}
    \maketitle

    \section*{1a. Cholesky Factorization Algorithm}
    \fbox{ My full Cholesky Factorization \texttt{.jl} file can be found on Github at the following link: \href{https://github.com/TommySimcox/Math466Cholesky/blob/main/chol.jl}{Link to my GitHub} }\bigbreak
    My algorithm is written in Julia, and consists of just one function called \texttt{chol()}, which calls one auxilary
    function I wrote called \texttt{isPositiveDefinite()} which returns a \texttt{Bool} based on whether or not a passed in matrix 
    is postive-definite. Below is a code snippet of my \texttt{chol()} function:
    \begin{Verbatim}[fontsize=\small,xleftmargin=0cm]
        function chol(M::Matrix{Int64})::Union{Matrix{Float64}, Nothing}
           # if the matrix isn't positive definite, return log indicating so
           !isPositiveDefinite(M) && return println("Matrix is not positive definite.")
          
           # destructure dims of array into variables
           (m, n) = size(M)
           
           # create empty matrix consisting of floats
           U = zeros(Float64, m, n)
           j = 1
          
           # calculate Cholesky decomposition
           U[j, j] = sqrt(M[j, j])
           for col in eachcol(M)
              for (i, v) in enumerate(col)
                 # this is just a series of if checks that determines the calculation we need to do based on
                 # whether i < j, i > j, or i === j
                 i > j ? U[i, j] = v / U[j, j] :
                 i < j ? U[i, j] = 0 :
                 i === j && i === 2 ?
                 U[i, j] = sqrt(v - (U[i, j-1])^2) :
                 ""
              end
              j += 1
           end
           # return our new matrix U in upper-triangular form
           z = U[1, 2]
           U[1, 2] = U[2, 1]
           U[2, 1] = z
           return U
       end
    \end{Verbatim}

    As shown in the function's signature, I decided to take advantage of Julia's type narrowing capabilities and restricted the 
    function's argument to be of type \texttt{Matrix{Int64}}. Similarly, I made sure that the function will return a \texttt{Matrix{Float64}}, or \texttt{Nothing}.
    I specified this union return type because if the passed in matrix is not positive definite, I return a \texttt{prinln()}, and otherwise,
    I return the matrix factorization \texttt{U}.

    \newpage
    \section*{1b. Testing}
    Test matrix: $\begin{bmatrix}
        1 & 0\\
        0 & 4
    \end{bmatrix}$

    \bigbreak
    \noindent Julia REPL:
    \begin{Verbatim}
        julia> include("chol.jl")
        2×2 Matrix{Float64}:
         1.0  0.0
         0.0  2.0
    \end{Verbatim}

    \hrule\bigbreak
    \noindent Test matrix: $\begin{bmatrix}
        1 & 0\\
        0 & 0
    \end{bmatrix}$
    \bigbreak
    \noindent Julia REPL:

    \begin{Verbatim}
        julia> include("chol.jl")
        Matrix is not positive definite.
    \end{Verbatim}

    \hrule\bigbreak
    \noindent Test matrix: $\begin{bmatrix}
        2 & 1\\
        1 & 2
    \end{bmatrix}$
    \bigbreak
    \noindent Julia REPL:

    \begin{Verbatim}
        julia> include("chol.jl")
        2×2 Matrix{Float64}:
         1.41421   0.707107
         0.0       1.22474
    \end{Verbatim}

    \hrule\bigbreak
    \noindent Test matrix: $\begin{bmatrix}
        2 & 1\\
        3 & 4
    \end{bmatrix}$
    \bigbreak
    \noindent Julia REPL:
    \begin{Verbatim}
        julia> include("chol.jl")
        Matrix is not positive definite.
    \end{Verbatim}
    \hrule\bigbreak
    \noindent Test matrix: $\begin{bmatrix}
        1 & 2\\
        2 & 3
    \end{bmatrix}$
    \bigbreak
    \noindent Julia REPL:
    \begin{Verbatim}
        julia> include("chol.jl")
        Matrix is not positive definite.
    \end{Verbatim}
    \hrule\bigbreak
    \noindent Test matrix: $\begin{bmatrix}
        5 & -1\\
        -1 & 3
    \end{bmatrix}$
    \bigbreak
    \noindent Julia REPL:
    \begin{Verbatim}
        julia> include("chol.jl")
        2×2 Matrix{Float64}:
          2.23607   -0.447214 
          0.0       1.67332
    \end{Verbatim}
    \newpage
    \section*{1c. \& 1d.}
    Running the commands \texttt{z.L * z.U - B} and \texttt{norm(z.L * z.U - B)} resulted in the following matrix and float, respectively:
    \begin{Verbatim}[xleftmargin=4cm]
        julia> z.L * z.U - B
        3×3 Matrix{Float64}:
         0.0  0.0  0.0
         0.0  0.0  0.0
         0.0  0.0  0.0

         julia> norm(z.L * z.U - b)
         0.0
    \end{Verbatim}
    Both of these results make sense, and show that the factorization was accurate. Had there been any rounding or truncating error, we would 
    have seen it in the above calculations.



    \section*{1e. Julia REPL Outputs For Matrices in \texttt{prog1c.dat}}
    \subsection*{(Skipping ones that are not pos-def per instructions)}
    \bigbreak
    I hard-coded the matrices in the above file, but here's the method I wrote for displaying the result of the \texttt{cholesky()} 
    method for each big matrix (if it exists):
    \begin{Verbatim}
    # container for big matrices in prog1.dat
    bigMatrices = [D_1, D_2, D_3, D_4, D_5, D_6, D_7, D_8]
       
    # if matrix in bigMatrices has a cholesky decomp (i.e. doesn't throw), display it
    map(x -> try display(cholesky(x)) catch e println("Matrix not pos-def.") end, bigMatrices);
    \end{Verbatim}
    \bigbreak
    And here's the resulting REPL output:
    \bigbreak
    \begin{Verbatim}
    julia> include("chol.jl")
    Cholesky{Float64, Matrix{Float64}}
    U factor:
    3×3 UpperTriangular{Float64, Matrix{Float64}}:
     3.74166  8.55236  14.1648
      .       1.96396   3.49149
      .        .        0.408248

    Matrix not pos-def.

    Cholesky{Float64, Matrix{Float64}}
    U factor:
    4×4 UpperTriangular{Float64, Matrix{Float64}}:
     1.0  2.0  3.0   4.0
      .   5.0  6.0   7.0
      .    .   8.0   9.0
      .    .    .   10.0

    Cholesky{Float64, Matrix{Float64}}
    U factor:
    5×5 UpperTriangular{Float64, Matrix{Float64}}:
     1.22491  0.86684   1.12125    0.616258  0.643779
      .       0.840492  0.522972   0.410345  0.774078
      .        .        0.794623  -0.137927  0.0562672
      .        .         .         0.407503  0.082128
      .        .         .          .        0.293004

    Matrix not pos-def.

    Matrix not pos-def.
    
    Matrix not pos-def.

    Cholesky{Float64, Matrix{Float64}}
    U factor:
    6×6 UpperTriangular{Float64, Matrix{Float64}}:
     0.562231  0.573649   0.0814434   0.226344   -0.223845  -0.0343774
      .        0.500819  -0.358015   -0.0600356   0.253891   0.491001
      .         .         0.787863   -0.115487   -0.294512  -0.384284
      .         .          .          0.67844    -0.417802   0.42769
      .         .          .           .          0.616474  -0.174615
      .         .          .           .           .         0.237077
    \end{Verbatim}
\end{document}
